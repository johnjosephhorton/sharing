\subsection{Platform's incentive to reduce bringing-to-market costs}
The total transaction volume in the P2P rental market is $Q_{BTM} r_{BTM}$.
If the platform creating the P2P rental market takes revenue proportional to this market volume, what is their incentive to reduce BTM costs?
On the one hand, lowering these costs raises the quantity transacted, but it also lowers the price.
If we assume that profits are proportional to the total transaction volume, then 
\begin{align}
  \frac{\partial \pi}{\partial \gamma} \propto \theta(1-\theta)(\alpha_L + \gamma \theta) > 0. 
\end{align}
For any equilibrium, slightly raising BTM costs \emph{raise} profits, assuming the P2P rental market still exists.
However, if the firm instead has revenue proportional to the total social surplus it generates, then incentives change. 
The surplus of owners is now
\begin{align}
  v_H = \alpha_H^2 - \frac{(r-\gamma)^2}{4} - (1 - \alpha_H)\gamma - (r-\gamma)\gamma/2, 
\end{align}
while the surplus for the non-owners is still $\alpha_L^2 - r^2/4$. 
The total surplus is
\begin{align}
  S = (1 - \theta)\alpha_L^2 - \frac{(r - \gamma)^2}{4}
\end{align} 
and
\begin{align}
  \frac{\partial S}{\partial \gamma} = (1 - \theta)\alpha_L^2 - \frac{(r - \gamma)^2}{4}
\end{align} 
