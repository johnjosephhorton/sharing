% To keep the analysis simple, we initially ignore transaction costs. 
% We modify the basic framework to include these costs and consider how it changes the results. 
% One of the main findings is that when transactions costs are sufficiently high, the P2P rental market cannot exist. 
% This finding suggests that the technological changes---namely the maturation and increasing penetration of the Internet and web-based technologies---were the technological shock that made these P2P rental markets feasible. 

In practice, the utilization of a good---even with an efficient P2P rental market---will be far less than 100\%.
Setting up trades, making repairs, transporting goods and so on all take time.  
Furthermore, even durable goods are consumed more quickly when used more intensively. 
There are several ways one could model these kinds of practicalities. 

For utilization, one modeling approach is to simply re-define what is the unit of time available and the corresponding $\alpha$. 
For example, we might think of a the unit of time for a vacation home on a ski slope to be 4 months, with high-types wanting to take three week vacations and low-types one week vacations and one week in total lost to cleaning and maintenance.      
For transaction costs, we could think of owners in the P2P rental market as facing a cost of $c$ that captures both the transaction costs of listing on a market, finding trading partners and so on, as well as the cost from increased usage that leads to either more extensive or more frequent repairs or faster replacement.
In the short-run P2P equilibrium, $c$ provides a price-floor in the rental market.  
In the long-run P2P equilibrium, the rental rate ``includes'' these costs, with $r_{LR} = p + c$. 

If $c$ is sufficiently high, then no P2P rental market will exist in either the short- or long-run: 
many goods have ``missing'' rental markets because the rental rate $r$ required to cover the added transaction costs of renting would be prohibitive.
Goods that have unpredictable usage patterns would be particularly poor rental candidates.  
It is only with the emergence of computer-mediated platforms that seem to dramatically reduce transaction costs that a P2P rental market has emerged for some of these goods. 
Before these markets sprung up, simply finding an appropriate trading partner would be difficult, to say nothing of coming to terms, writing a contract, monitoring compliance, handling disputes, making payment and so on. 



Monitoring usage 
