
Now we assume that to bring any amount of supply to the market, the owner has to pay some cost.
This could be the cost of labor (as in the case of Uber) for a good that requires a labor input or it could just be the time cost of finding a renter for the offered good and the time it takes to facilitate a transaction.  
We assume that this cost for an individual $i$ is $\gamma_i$ and that it does not depend on the amount brought to the market.
This cost is independent and indentically distributed across owners, with $\gamma \sim F$.

If an owner stays out of the market, they get $\alpha^2$ in consumption utility.
If they rent, they lose $r^2/4$ in foregone consumption, get $r (1 - (\alpha_H - r/2))$ in rental income and pay the transaction cost, $\gamma$.
The owner just indifferent between renting out and not renting has a $\hat{\gamma}$ such that
\begin{align} \label{eq:indiff}
  \hat{\gamma} & = r (1 - (\alpha_H - r/2)) - r^2/4 \\
               & =  (1-\alpha_H)r - r^2/2 
\end{align}
All owners will a $\gamma > \gamma_i$ will stay out of the market, while those with a $\gamma < \gamma_i$ will rent out.
Market clearing now requires that
\begin{align}
  \theta F(\hat{\gamma})(1 - x_H(r)) = (1-\theta)x_L(r), 
\end{align} 
with the marginal owner being indifferent condition from Equation~\ref{eq:indiff}. 
$F(\hat{\gamma})$ is the fraction of owners with transaction costs lower than that of the indifferent owner. 

When $r$ is higher, two forces bring more supply into the market:
(1) greater economization by owners already providing the good in the market
(2) more owners willing to participate.

Lower transaction costs bring more supply into the market, as more owners find it profitable to participate.
However, this increased in supply on the extensive margin also lowers rental rates, which will reduce economization of owner-usage but increase usage by non-owners.
In the long-run, the ideal owners are those with low transaction costs rather than those with a high valuation of the good.

What are some questions/thoughts?
\begin{itemize}
\item Does an equilibrium always exist? Presumably if transaction costs are too high, the only the supportable equilibium would be one with a high $r$, but if $r > \alpha_L$, there is no demand. This would work nicely with out ``tech lowered transaction costs'' story. 
\item How should shocks to transaction costs be modeled? I tried assuming uniformly distributed transaction costs with a multiplier parameter. However, even with this extreme simplification, the equilibrium $\gamma$ and $r$ are very, very complex. I imagine we could still do comparative statics, but it would be a bit tricker than what we have now, where the equilibrium is a nice closed form solution obtained through simple algebra. Perhaps there is some easier way to model transaction costs that gets us back to that world. 
\item In keeping with the theme of extreme simplicity, I could just assume two transaction cost types, $\gamma_H$ and $\gamma_L = 0$ that are orthogonal to valuation and then look at the resulting equilibria.
\item What is the deadweight loss of transaction costs?
  \item What is the incidence of transaction costs? Taxation? 
\end{itemize} 

