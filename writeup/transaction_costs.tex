
Now we assume that to bring any amount of supply to the market, the owner has to pay a cost.
This could be the cost of labor for a good that requires a labor input, such as in the case of driving with Uber or cleaning up an apartment when hosting on Airbnb.
It could alo consists of conventional transaction costs, such as finding a counter-party, coming to terms, monitoring the contract and so on. 
We assume that these costs differ by individual.
For example, some people enjoy driving, talking to strangers and have a full-time job with fixed and inflexible hours.
For them, driving for Uber is very low cost and may be something close to a hobby (though they still bear gas and depreciation costs).

We assume that the cost of bringing any supply to the P2P rental market for an individual $i$ is $\gamma_i$.
This cost is independent and indentically distributed across the population, with $\gamma \sim F$.

As we would expect, owners with particularly low value for $\gamma$ are the ones that supply to the rental market. 
If an owner stays out of the market, they get $\alpha^2$ in consumption utility.
If they rent, they lose $r^2/4$ in foregone consumption, get $r (1 - (\alpha_H - r/2))$ in rental income and pay the cost, $\gamma$.
The owner just indifferent between renting out and not renting has a $\hat{\gamma}$ such that
\begin{align} \label{eq:indiff}
  \hat{\gamma} & = r (1 - (\alpha_H - r/2)) - r^2/4 \nonumber \\
               & =  (1-\alpha_H)r - r^2/2 
\end{align}

All owners will a $\gamma > \gamma_i$ will stay out of the market, while those with a $\gamma < \gamma_i$ will rent out.
Market clearing in quantities now requires that
\begin{align}
  \theta F(\hat{\gamma})(1 - x_H(r)) = (1-\theta)x_L(r). 
\end{align} 
This condition, combined with the definition of the transaction cost for the marginal owner, defines an equilibrium with $r$ and $\hat{\gamma}$.

\subsection{How do transaction costs affect the short-run equilibrium?}
When $r$ is higher, two forces bring more supply into the market:
(1) greater economization by owners already providing the good in the market
(2) more owners willing to participate in the P2P rental market.

\subsection{The indcidence of transaction costs} 
If transaction costs are lowered, who benefits?
Owners have lower costs, but some of this gets passed back to non-owners in the form of a lower rental rate. 

\subsection{How do transaction costs affect the long-run equilibrium?} 
With the existence of transacton costs,

Lower transaction costs bring more supply into the market, as more owners find it profitable to participate.
However, this increased in supply on the extensive margin also lowers rental rates, which will reduce economization of owner-usage but increase usage by non-owners.
In the long-run, the ideal owners are those with low transaction costs rather than those with a high valuation of the good.

What are some questions/thoughts?
\begin{itemize}
\item Does an equilibrium always exist? Presumably if transaction costs are too high, the only the supportable equilibium would be one with a high $r$, but if $r > \alpha_L$, there is no demand. This would work nicely with out ``tech lowered transaction costs'' story. 
\item How should shocks to transaction costs be modeled? I tried assuming uniformly distributed transaction costs with a multiplier parameter. However, even with this extreme simplification, the equilibrium $\gamma$ and $r$ are very, very complex. I imagine we could still do comparative statics, but it would be a bit tricker than what we have now, where the equilibrium is a nice closed form solution obtained through simple algebra. Perhaps there is some easier way to model transaction costs that gets us back to that world. 
\item In keeping with the theme of extreme simplicity, I could just assume two transaction cost types, $\gamma_H$ and $\gamma_L = 0$ that are orthogonal to valuation and then look at the resulting equilibria.
\item What is the deadweight loss of transaction costs?
  \item What is the incidence of transaction costs? Taxation? 
\end{itemize} 

